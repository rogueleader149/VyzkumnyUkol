\cleardoublepage\phantomsection   % cleardoublepage a phantomsection udělají že odkaz z obsahu vede NAD nadpis, jinak vede POD
\addcontentsline{toc}{chapter}{Conclusion}
\chapter*{Conclusion}

After an introduction to DNA computation, formal languages and complexity theory in Sections \ref{sec:DNA}, \ref{sec:minf}, we summarized known results about the relation between strand models and Chomsky hierarchy in Section \ref{sec:strand}.

In Section \ref{sec:wang} we focused on Wang tile models and presented four kinds of resources for DNA computations. Then we studied the relation between these resources and ``classical'' resources of Turing machine -- time and space. In Section \ref{sec:my_TU} we presented a new tilesystem which directly simulates a Turing machine at $\tau = 2$ in 2D. Thus some relations between those resource consumptions could have been proved. We also stated reasonable conditions for a feasible DNA algorithm and as a consequence we proved that $\BPP$ is feasible at $\tau = 2$ in 2D.

In Chapter \ref{chap:problems} we derived a model suitable for solving $\NP$ problems. Using this model we presented DNA tilesystems which solve hard computational problems: $k$-clique Problem, Graph 3-coloring Problem and Graph Isomorphism Problem. These tilesystems were supplied by studied complexities.

In Section \ref{sec:xgrow} we described an implementation of our $k$-clique algorithm in open source $\atam$ simulator {\tt xgrow}.

Another interesting problem which arised during the study of Turing universality is $n\times n$ Squares Problem described within Section \ref{sec:wang_power}. This problem might be a direction of future research.

% -- we introduced four kinds of resources for DNA computations
% -- we presented a new tileset which directly simulates TM at $\tau = 2$ in 2D
% -- we proved some relations between resource consumption of TM and the tileset
% -- we stated conditions for feasibility
% -- we proved that classical BPP complies these conditions
% -- we derived a model suitable for solving $\NP$ problems
% -- the model directly says which tiles are to be prepared
% -- we proposed algorithms for $k$-clique, Graph $3$-coloring and Graph Isomorphism
% -- and computed their complexities
% -- we performed a simulation in xgrow -- an open source aTAM/kTAM simulator
% -- => we connected theoretical complexity and aTAM at $\tau = 2$ in 2D
% future/further research: $n\times n$ squares vs. Co(2D)W, periodic properties of 2D words vs. Wang tiling

% another topics: self-assembly of $n\times n$ squares
