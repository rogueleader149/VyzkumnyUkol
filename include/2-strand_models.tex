\chapter{Strand models}

% úvod o kapitole, přehled

Quick overview of considered structures. Winfree's overview (pg 29 -- considered molecules, pg 36 -- sizes of DAE and a better picture, pg 37 -- comparison of DAO/DAE in a lattice, explanation pg 43).

Seeman, Fu and their DAO/DAE in \cite{seeman93}, is the picture of DAO strange?

\section{Adleman's experiment}
	
	Adleman showed in his ground-breaking work, \cite{adleman94}, that DNA molecules are really capable of computation. He exploited that huge parallelism possible in DNA computation for one of the most fundamental NP-complete problems -- the Hamiltonian Path Problem (HPP) in directed graph with designated vertices $v_{begin}$ and $v_{end}$.
	
	Let us remind this type of HPP. Given a directed graph $G_n$ with $n$ vertices and two designated vertices $v_{begin}$ and $v_{end}$, the problem is to answer whether there exists an oriented path from $v_{begin}$ to $v_{end}$ through the graph such that the path visits every vertex. Note that {\em path} cannot visit any vertex more than once from definition.
	
	\begin{figure}[H]
	\begin{center}
		\includegraphics{./figures/adleman_graph.pdf}
		\caption{Adleman's original graph.}
		\label{fig:adleman_graph}
	\end{center}
	\end{figure}
	
	Adleman originally used a graph with seven vertices shown in figure \ref{fig:adleman_graph}. It can be seen that the path $0 \rightarrow 1 \rightarrow 2 \rightarrow 3 \rightarrow 4 \rightarrow 5 \rightarrow 6$ is Hamiltonian\footnote{Note that it can be re-labelled such a nice way without loss of generality.}.
	
	Adleman first presents this non-deterministic five-step algorithm, whose steps are then described in terms of DNA manipulations:
	\begin{description}
		\item[Step 1] Generate random paths through the graph.
		\item[Step 2] Keep only those paths that begin with $v_{begin}$ and end with $v_{end}$.
		\item[Step 3] If the graph has $n$ vertices, then keep only those paths that enter exactly $n$ vertices.
		\item[Step 4] Keep only those paths that enter all of the vertices of the graph at least once.
		\item[Step 5] If any paths remain, say ``Yes''; otherwise, say ``No.''\footnote{This is the original version, I would rectify the fifth step: If any paths remain, say ``Yes''; otherwise, say ``{\em I do not know.}'' That is because NP problem gives answer ``Yes'' iff there {\em exists} supporting solution. To say ``No'' one needs to show that {\em all} solutions do not satisfy. That is exactly the difference between NP and co-NP.}
	\end{description}
	To see % sloveso od insight ?
	how DNA can compute, let us describe this example more precisely. The computation itself (I mean the inception of the final solution) is hidden in Step 1. Each vertex $i$ is associated with a random\footnote{We will expect those sequences to be different enough.} $20$-mer sequence of DNA, let us denote its $5'\rightarrow 3'$ orientation by $O_i$, its 10-mer prefix by $p_i$ and its 10-mer suffix by $q_i$. Each edge $i\rightarrow j$ is then associated with $\overline{q_i p_j}$ sequence with reverse backbone orientation ($3'\rightarrow 5'$) where $\overline{q_i}$ stands for Watson-Crick complementary word. There is an exception for $i=begin$ and $j=end$: instead of $\overline{q_{begin} p_j}$ there is $\overline{O_{begin} p_j}$ and in a similar way for $j=end$.
	
	\begin{figure}[H]
	\begin{center}
		\includegraphics{./figures/adleman_strands.pdf}
		\caption{Example of assigned sequences.}
		\label{fig:adleman_strands}
	\end{center}
	\end{figure}
	
	It can be easily seen that all correctly bonded double-strands correspond with a valid path in $G_n$. Moreover, all complete double-strands represent a valid path from $v_{begin}$ to $v_{end}$ through $G_n$.
	
	All the other steps are fully described in \cite{adleman94}. The most important thing is that the most time-demanding step is Step 4. In this step one has to purify the product of Step 3 with a biotin-avidin magnetic beads system. This process extracts consequently for every vertex $i$ only those DNA strands which contain a substring representing vertex $i$. Thus its biostep complexity is $O(n)$. If we assume that one biostep takes at least tens of minutes and it should be performed repeatedly to avoid errors, we can conclude that $O(n)$ is just too much\footnote{Winfree, \cite{winfree_phd}, gives a positive solution.}.

\section{Single-stranded molecules}
	
	SAT in $O(1)$ biosteps etc.

\section{Double-stranded molecules}
	
	\subsection{Linear strands}
		
		Equivalent to regular languages.
	
	\subsection{Dendrimer structures}
		
		Equivalent to context-free languages.

\section{Double crossover molecules}
	
	Equivalent to recursively enumerable languages (Turing universal). Important notes in 3.2.5 Winfree -- single side hybridization -- how to avoid. Tricky solution of Hamiltonian Path Problem.