\cleardoublepage\phantomsection   % cleardoublepage a phantomsection udělají že odkaz z obsahu vede NAD nadpis, jinak vede POD
\addcontentsline{toc}{chapter}{Prolog}
\chapter*{Prolog}

	1959: Feynman's visionary talk, \cite{feynman}; 
	
	The ground-breaking work was carried out by Adleman, \cite{adleman94}, who showed that DNA computation is practically feasible. In his experiment, Adleman used special DNA sequences for solving Hamiltonian Path Problem, one of the most typical NP-complete problems.
	
	... Extreme parallelism! But also possibility of errors.

\section*{Work overview}
	
	%%%%%%%%%%%%%%%%%%%%%%%%%%%%%%%%%%%%%%%%%%%%%%%%%%%%%%%%%%%%%%%%%%%%%%
	
	Chapter 1: Intro.\\
	Chapter 2: First of all I will describe models which exploit specific DNA structure.\\
	Chapter 3: Abstract Tile Assembly Model, temperature, 2D vs. 3D.\\
	Positive integers $\N$. \nomenclature{$\N$}{Positive integers.}
	
	% zkusíme Ramseyovo číslo R(5,5) ? :)
	% -> problém "Je R(n,n) větší než něco?" znamená pro odpověď ANO najít graf t.ž. neobsahuje K_n ani anti-K_n .. což je samo o sobě co-NP (neboli to je \exists \forall stroj .. \Sigma_2 jazyk, horní hranice pak je \Pi_2 jazyk)
	% 
	% NP vs co-NP .. co-NP přece nedává ne?