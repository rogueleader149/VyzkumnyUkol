\cleardoublepage\phantomsection   % cleardoublepage a phantomsection udělají že odkaz z obsahu vede NAD nadpis, jinak vede POD
\addcontentsline{toc}{chapter}{Preface}
\chapter*{Preface}

In 1959 Feynman gave a visionary talk {\em There's plenty of room at the bottom} \cite{feynman} where he anticipates that future computers will work in molecular scales. Many years later in 1994 Adleman conducted a real experiment \cite{adleman94} where he used DNA molecules to solve an $\NPC$ problem -- Hamiltonian Path Problem. DNA computation became popular because of promising advantage in solving hard computational problems which is extreme parallelism.

Unlike initial excitement it showed to be very hard to use for practical problems, DNA computations still do too many errors %!% citovat ten článek z Nature
Many researchers are trying hard to decrease the amount of errors, many others are studying DNA computations from mathematical informatics point of view.

This work is dedicated especially to the mathematical informatics aspects arising from models of DNA computation.

\section*{Work overview}
	
	\begin{description}
		\item[Chapter \ref{chap:intro}.]
		%!% DNA, languages from resource comsumption point of view, strand and Wang tile models; we added studied complexities, straightforward simulation of TM => we can compare resource comsumption; BPP preserves
		\item[Chapter \ref{chap:problems}.]
		%!% based on Winfree's algorithm we present a simplified model based on Wang tiles and give solution to three hard computational problems
	\end{description}

\section*{Notation and used symbols}
	
	%!% bude vopruz
	
	%%%%%%%%%%%%%%%%%%%%%%%%%%%%%%%%%%%%%%%%%%%%%%%%%%%%%%%%%%%%%%%%%%%%%%
	
	% zkusíme Ramseyovo číslo R(5,5) ? :)
	% -> problém "Je R(n,n) větší než něco?" znamená pro odpověď ANO najít graf t.ž. neobsahuje K_n ani anti-K_n .. což je samo o sobě co-NP (neboli to je \exists \forall stroj .. \Sigma_2 jazyk, horní hranice pak je \Pi_2 jazyk)
	