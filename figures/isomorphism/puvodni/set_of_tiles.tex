\begin{description}
	\item[Bottom tiles.] These tiles have almost the same rules as in graph 3-coloring, the difference is that all one-color combinations are omitted. $\frac{n}{2} \cdot n(n-1) \sim \frac{n^3}{2}$ tile types were required.
	\item[Bottom corner tiles.] These tiles are exactly the same like for $k$-clique. $2$ tile types were required.
	\item[Inner tiles.] There are three types of inner tiles:
	\begin{description}
		\item[Number-ordering tiles.] These are similar to previous ones, the difference is which do exist and which do not. Let us assume a tile with numbers $k$ and $l$ with colors $a$ and $b$, respectively. Note that numbers $k$ and $l$ correspond with vertices in graph $G$ and colors $a$ and $b$ correspond with vertices in graph $H$. This tile must verify the isomorphism property -- existence or non-existence of edge between appropriate vertices. Thus the tile exists iff
		$$(\{k,\,l\} \in E(G) \wedge \{a,\,b\} \in E(H)) \vee (\{k,\,l\} \notin E(G) \wedge \{a,\,b\} \notin E(H)) . $$
		For similar reasons all pairs of vertices from graph $G$ meet each other, thus every edge is verified so the condition $(2)$ is satisfied. $4e^2 + 4\bigl(\binom{n}{2}-e\bigr)^2 + 2n(n-1) \sim 8e^2 - 4en^2 + n^4$ tile types were required.
		\item[Color-extracting tiles.] Now we have to extract colors (forget numbers) and order them in given order so that we can verify that every color is used exactly once. This process will be triggered by a special inner tile with the highest number of arbitrary color and a non-colored asterisk on the bottom. On the top it will have an asterisk of that number's color and a non-colored asterisk. For every other number with arbitrary color there exists a tile with it and an asterisk of an arbitrary but different color on the bottom. On the top it will have two asterisks of these colors in correct order. $n + n(n-1)^2 \sim n^3$ tile types were required.
		\item[Color-ordering tiles.] These are similar to those with numbers. Similarly there do not exist tiles with one color. $n^2 - n + (n-1) \sim n^2$ tile types were required.
		% there is some redundancy .. but no idea how to do it faster .. it can be done easier, one can only verify without colors, same colors are killed during ordering! %!% mam tady pravdu => předělat, bude to mnohem menší !!!
	\end{description}
	\item[Border tiles.] These tiles are exactly the same like for $k$-clique. $2$ tile types were required.
	\item[Verification tiles.] As soon as there appears a combination of sharp and most-left-colored asterisk, a verification tile comes having ``C'' of the second color on the right top. After this initialization there are tiles with colored ``C'' and same-colored asterisk on the bottom and next-colored ``C'' on the right top. This ensures that every color was used exactly once. The last color is followed by non-colored ``C''. $n$ tile types were required.
	\item[DONE tile.] Finally if non-colored ``C'' meets non-colored asterisk, a ``DONE'' tile is connected signalizing correct solution. $1$ tile type was required.
\end{description}

Summed up, this DNA algorithm requires $8e^2 - 4en^2 + n^4$ tile types. Binding complexity is $2\nicefrac{1}{2}\,n^2$, glue complexity is $n^2$.
